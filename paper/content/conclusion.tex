\phantomsection{Mon stage dans la start-up Amiral Technologies :}
En l'espace de 14 semaines, j'ai eu le privilège de rejoindre les rangs d'Amiral Technologies, une start-up dynamique évoluant dans le secteur des technologies de l'information.
Mon stage a été marqué par une variété de tâches significatives, chacune contribuant à ma croissance professionnelle et à l'évolution de l'entreprise.

Mon rôle s'est notamment concentré sur la refonte des fichiers utilisés lors de l'intégration continue de GitLab sur Docker, optimisant ainsi les processus de développement.
De plus, j'ai eu l'opportunité de sensibiliser mes collègues à la cybersécurité, apportant à mes collègues une nouvelle dimension à la cybersécurité.

Ce stage m'a permis de plonger dans un environnement d'apprentissage dynamique, où j'ai découvert de multiples technologies et où j'ai constaté une nette montée en compétences.
J'ai été particulièrement marqué par l'accueil chaleureux et la culture d'innovation qui règnent au sein de l'entreprise.

Bien que des défis aient émaillé mon parcours, tels que le changement de maître de stage et la nécessité de travailler en autonomie, ils m'ont offert l'occasion d'apprendre et de grandir.
J'en tire une appréciation accrue pour l'importance de la communication et de l'adaptabilité dans le monde professionnel.

En considérant ces opportunités, je me projette volontiers dans une entreprise à taille humaine, tout en gardant intacte ma curiosité pour explorer davantage le vaste monde du travail.
Mon stage chez Amiral Technologies a marqué mon parcours dans le monde professionnel, pavé de découvertes et de contributions significatives.
