\section{Amiral Technologies - une startup prometteuse}
\subsection{Description de l'entreprise}
Amiral Technologies est une startup se définissant comme un dérivé du CNRS. 
Société par actions simplifiée créée en 2018, elle est composée de 22 personnes et vend un logiciel de prédiction de pannes appelé diagfit.

\subsection{Une entreprise dans un marché exeigeant}
Reconnu pour la qualité du logiciel Diagfit, plusieurs clients de renom ont fait appel à l’expertise d’Amiral Technologies afin d'améliorer leur programme de maintenance de leurs équipement.


Amiral Technologies s'oriente vers trois secteurs précis pour canaliser leur méthode de communication : le secteur des transports avec les entreprises ferroviaires et aéronautiques, le secteur manufacturier en aidant le développement d'une industrie 4.0 et le secteur de l'énergie et du nucléaire.
\\
Un levier de compétitivité d'Amiral Technologies et de Diagfit est la méthode d'apprentissage préalable à la détection de pannes : celles-ci étant très couteuses, ne pas à avoir à les repriduire est nécessaire pour l'utilisation d'un outil de ce genre.
L'apprentissage en mode aveugle se traduit également par une polyvalence propre à l’entreprise. 
En effet, le problème souvent rencontré par les entreprises de détection de pannes est la nécessité de comprendre la réalité physique qui se cache derrière les données.
La force d’Amiral réside dans sa capacité à traiter une multitude d’équipements différents de la même manière, que ce soit une pompe à vide, un moteur d’avion ou un rail de train
\\
La détection de pannes permet d'entreprendre des maintenances prédictives en fonctionnement des données spécifiques à l'appareil en question et non à une moyenne.
Cette technologie permet alors de réduire les coûts de maintenance en consacrant les maintenances uniquement lorsqu'elles sont nécessaires plutôt que d'une manière périodique.

\subsection{Organisation interne}
L'entreprise est basée en Isère dans la ville de Grenoble et fait remarquer sa présence dans les salons de Lyon et Paris.
Les membres d'Amiral Technologies qui ont un pouvoir décisionnel sont Katia Hilal - présidente et co-fondatrice de l'entreprise avec Mazen Alamir, Simon Gazikian - directeur général et Sebastien Le Gall - directeur technique.
Les membres d'Amiral Technologies sont répartis en différents groupes : l'équipe commercial, les data-scientists, les développeurs interface utilisateurs [front end], les développeurs technique [backend] et les administrateurs système.
J'ai eu l'opportunité lors de ce stage de travailler avec les deux dernières équipes.

\todo{METTRE UN ORGANIGRAMME}

Amiral Technologies développe son Comité Social et Economique [CSE].
D'autre part, l'entreprise dit s'investir dans la responsabilité sociétale des entreprises [RSE] avec des changements qui s'avèrent inefficaces.
\\
De mon point de vue, les choix qui ont été faits pour la RSE n'ont pas été fait dans l'optique de vouloir améliorer le système et plus dans un sentiment d'obligation de faire des efforts.
On peut illustrer mes propos avec le fait que la majorité des mesures qui ont été prises sont portées vers le pilier écologique du développement durable, un moyen de faire des efforts d'une manière visible.
\\
On peut se poser la question de l'utilité de la mise en place d'outils digitaux de signature éléctonique qui force les documents à être sauvegardés plusieures fois plutôt que d'être imprimés une fois.
L’investigation de l’empreinte carbone de nos algorithmes et de notre stockage est un effort complètement inutile puisque les données qui en sont ressorties n'ont jamais été exploitées

Enfin, le compte rendu de la responsabilité sociétale d'Amiral Technologies proposait une technologie de stockage différente de celle utilisée lors de la présentation mais n'a jamais été mise en place à cause du coût de transition.






% \title{Présentation d'entreprise}

% \section{Introduction}
% \begin{frame}{Introduction}
%   \begin{itemize}
%     \item<1-> Startup Française dans la prédiction de pannes : \textbf{Amiral Technologies}.
%     \item<2-> Secteur d'activité : \textbf{Transport, Industrie et Énergétique}.
%     \item<3-> Taille et localisation : \textbf{vingt-cinq employés} située à \textbf{Grenoble, Isère}.
%     \item<4-> Aperçu général du stage :
%         \begin{itemize}
%         \item Durée : \textbf{14 semaines - 3 mois et demi}.
%         \item Département : \textbf{Administration Système Informatique}.
%         \item Objectifs du stage : \textbf{Assistance dans le maintien en condition opérationnelle de l'entreprise et renforcement de la cybersécurité}.
%         \end{itemize}
%   \end{itemize}
% \end{frame}


% \section{Amiral Technologies dans le temps}
% \subsection{Historique de l'entreprise}
% Amiral Technologies se définit comme une "spin-off" du Centre national de la recherche scientifique (CNRS). 
% Fondée en 2018 par le Dr Mazen Alamir, directeur de recherche au CNRS, et le Dr Katia Hilal, maintenant directrice des opérations de l'entreprise.
% Amiral Technologies a pour ambition de devenir une figure de proue de la maintenance prédictive. 
% Ses innovations sont le fruit de 10 années de recherche académique en théorie du contrôle, automatique et en apprentissage automatique.

% \subsection{L'outil DiagFit}
% La vision de l’entreprise est de permettre aux équipements industriels de fonctionner sans défaut et sans pannes tout en évitant les maintenances non nécessaires.
% Pour réaliser cet objectif, l’entreprise s’est donné pour mission de développer un logiciel de prédiction de pannes à partir de données de capteur.
% Le logiciel DiagFit utilise différents modèles d'apprentissage automatique (machine learning) reste exploitable par des experts métiers n’ayant aucune connaissance en science des données (data science).

% Reconnu pour sa qualité, plusieures grandes entreprises de renom commencent à inclure l'outil DiagFit dans leur systèmes :
% PLEIN DIMAGES PINGUUUUUUUUUU :D


% \section{L'avantage concurrenciel d'Amiral Technologies}
% logiciel peut traiter données de capteurs différents sans adaptation 
% logiciel peut travailler sur la détection de données saines uniquement pour fournir une prédictino de panne


% \section{Structure organisationnelle}
% \subsection{Départements clés et responsabilités}
% \begin{itemize}
%     \item \textbf{Président} - Hilal Katia
%     \item \textbf{Directeur Général} - Gazinkian Simon 
%     \item \textbf{Directeur Technique} - Le Gall Sébastien 
%     \item \textbf{Administrateur Système} - Delgutte Théo 
% \end{itemize}
% Aujourd’hui Amiral Technologies est une entreprise composés d’une vingtaines de personnes, répartis sur plusieurs sites et travaillant au sein d’équipes.
% On trouve notamment, trois grands pôles sur le site de Grenoble :
% — L’équipe Back-end : qui gère la partie "architecture" du logiciel
% — L’équipe Front-end : qui s’occupe de l’interface du logiciel
% — L’équipe Data Science : qui conçoit et programme le coeur algorithmique de Diagfit
% \begin{figure}[ht]
%     \centering
%     \includegraphics[width=0.8\textwidth]{organigramme.png}
%     \caption{Organigramme de l'entreprise}
%     \label{fig:organigramme}
%   \end{figure}

% \subsection{Méthode de travail}
% Amiral Technologies fonctionne en mode agile, en suivant la méthode Scrum : l’entreprise travaille en sprint de 4 à 6 semaines.
% Chaque sprint est composé de plusieurs épic (grandes fonctionnalités) regroupant des tâches.
% À la fin d’un sprint, une revue est effectuée, puis l’on repart sur un nouveau sprint.
% Entre deux sprints, une journée de "Cool Down" permet aux employés de travailler sur un thème ou une idée qu’ils ont eu et qui ne rentrait pas
% dans le cadre du sprint.

% \subsection{Administration Système}
% Le pôle administration système joue un rôle essentiel dans le maintien en condition opérationnelle des systèmes informatiques de l'entreprise.
% Il est responsable de la gestion des sauvegardes et de la récupération des données, assurant ainsi la disponibilité et l'intégrité des informations cruciales.
% De plus, il effectue également l'inventaire des équipements et des logiciels, ce qui permet de maintenir une visibilité précise sur les ressources technologiques de l'entreprise.
% Grâce à ces responsabilités, le pôle administration système contribue à assurer la continuité des activités, la sécurité des données et la performance des systèmes, ce qui est essentiel pour la productivité et la réussite globale de l'organisation.

% Le pôle d'administrateurs système est composé d'une seule personne, Delgutte Théo.
% J'ai eu l'occasion de l'assister dans sa mission continue de maintien en condition opérationnelle de la partie informatique de l'entreprise.
