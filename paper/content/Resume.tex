\imtnechapitre{Résumés et Mots clés}
\section{Résumé en Français}
J'ai effectué un stage de 16 semaines à l'AFNIC (18 Millions d'euros de CA en 2021 et 91 employés), le registre français responsable de la gestion des noms de domaines finissant notement en .fr, .ovh, .paris et autre.
Dans le contexte de ce stage j'ai intégré l'équipe Labs principalement responsable de la R\verb|&|D.

L'objet principal de ce stage a été la conception et mise en place d'un système de monitoring afin de pouvoir avoir des données sur l'impact énergétique du DNS.
Pour cela j'ai modifié le docker-compose qui constitue le resolveur utilisé en interne afin d'activer les différentes options de monitoring, d'intégrer les outils nécéssaires et d'ouvrir les différents flux, j'ai ensuite récupéré les données avec Prometheus avant de créer des graphiques sur Grafana.

Une autre mission que j'ai eu était de trouver un moyen de relier tous les Wattmetres connectés aux serveurs de production au serveur Grafana interne afin de pouvoir centraliser les données 

\section{Mots Cléfs}
DNS, Monitoring, Docker, Grafana, Prometheus, Scaphandre, RSE
\section{Executive Summary in English}
\todo{Quand Résumé en FR validé}
\section{Key Words}
DNS, Monitoring, Docker, Grafana, Prometheus, Scaphandre, CSR
