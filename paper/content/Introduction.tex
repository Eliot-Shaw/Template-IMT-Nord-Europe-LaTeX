\imtnechapitre{Introduction}
Après ma 3\textsuperscript{ème} à l'IMT (Institut Mines Telecom) Nord Europe anciennement IMT Lille Douai, j'ai effectué un stage de découverte de dommaine à l'AFNIC.

Ce stage s'est déroulé du 15 Mai au 31 Aout (soit 16 semaines) principalement dans les locaux de l'entreprise a Saint-Quentin-En-Yvelines (en périphérie de Paris) sous la responsabilité de Benoit Ampeau, Directeur Partenariat et Innovation et mon tuteur de stage.

Le sujet de ce stage était de concevoir et développer un système permétant de monitorer les différentes informations liés a un résolveur DNS dont les plus intéréssantes sont la consommation éléctrique en temps réel du serveur et la quantité de requêtes traités par le serveur.
Le but de cette mission est de pouvoir développer la connaissance de l'impact énergétique du DNS ce qui est nécéssaire pour la politique RSE de l'association et servira de base à des papiers ou présentations destinnées aux instances internationnales de la gouvernance et du développement de l'internet.

Pour pouvoir mener a bien cette mission, j'avais a disposition :
\begin{itemize}
    \item L'aide des autres membres de l'équipe Labs dont je faisait partie qui maitrisent l'architecture informatique interne et qui ont donc pus m'aider a choisir et implémenter les différentes solutions.
    \item Un accès sudo (administrateur) à tous les serveurs Labs
    \item Un accès a toute la documentation, au Gitlab et aux autres outils internes a l'AFNIC
\end{itemize}

