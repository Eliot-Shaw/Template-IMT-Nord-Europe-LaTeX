\section{Réflexions et perspectives personnelles}
\subsection{Impressions suite au changement de maître de stage}
\subsubsection{Gestion d'une situation délicate}
La transition vers un nouveau maître de stage a engendré un sentiment de désagrément.
Cette situation a présenté des défis sociaux, exigeant une gestion prudente des interactions avec les autres membres de l'équipe pour mieux comprendre cette transition.


\subsubsection{Altération de la continuité des missions}
La modification de mon encadrement a également eu un impact sur la continuité de mes missions.
Cette transition a entraîné un éloignement des projets à long terme au profit d'une série de tâches plus courtes.
Cette nouvelle orientation a suscité un moindre sentiment d'accomplissement professionnel.


\subsection{Environnement de travail et relations humaines}
\subsubsection{Attraits des Petites et Moyennes Entreprises (PME)}
Je considère avec faveur les expériences au sein de PME, principalement en raison des relations humaines plus chaleureuses et d'une communication directe avec la hiérarchie.
J'apprécie également l'absence d'une pression excessive en matière de performance, ainsi que la pertinence des critères d'évaluation alignés avec la réalité.


\subsubsection{Mon opinion sur l'openspace et le télétravail}
Bien que l'openspace ne corresponde pas à ma préférence personnelle, j'ai su m'adapter à cet environnement.
Cependant, le sentiment de surveillance constante et le manque d'intimité m'ont quelque peu dérangé.
En revanche, je reconnais les avantages de l'openspace pour favoriser la communication entre les collègues.

En ce qui concerne le télétravail, je le considère avec réserve.
Les contraintes liées à mon espace de travail personnel non approprié, la communication moins fluide et la perte de motivation sont des aspects qui en limitent l'efficacité.
De plus, je m'oppose idéologiquement à mélanger les espaces de travail, qui engendrent un stress inhérent, avec des espaces de détente tels que la chambre ou la cuisine.

\subsubsection{Apprentissage des expériences de camarades stagiaires}
J'ai été enrichi par les récits d'autres stagiaires dans mon entourage professionnel.
Certains d'entre eux ont vécu des stages compliqués en raison de superviseurs autoritaires ou de conditions de travail difficiles.
Ces témoignages m'ont sensibilisé à l'importance d'un environnement de travail favorable.
J'ai la ferme intention de tirer des enseignements de leurs expériences pour prendre des décisions judicieuses et éviter les écueils.


\section{Orientations Futures}
\subsection{Préférence pour les Environnements de Travail des PME}
J'entretiens un attachement particulier pour les environnements de travail au sein de PME.
Cependant, je suis ouvert à l'idée d'explorer une grande entreprise lors de mon stage de M1, dans le but d'acquérir une perspective équilibrée et d'élargir mes horizons professionnels.

\subsection{Favoriser les Projets de Long Terme}
Je trouve une plus grande satisfaction dans l'engagement envers des projets de long terme, par opposition aux missions ponctuelles dénuées de fil conducteur.

\subsection{Dilemme entre Développement et DevOps}
Actuellement, je me trouve à un carrefour entre deux voies professionnelles : le développement pur et le rôle de DevOps, un développeur avec une casquette d'administrateur système.
Mon expérience, initialement orientée vers l'administration système avant de se transformer en développement, m'a permis d'explorer ces deux domaines.

D'une part, le développement me séduit par sa créativité et la concrétisation d'idées innovantes.
De l'autre, le DevOps attire mon attention avec son approche holistique, fusionnant compétences techniques et gestion opérationnelle pour assurer la stabilité des systèmes.

Mes réflexions actuelles, guidées par mon expérience, joueront un rôle essentiel dans ma prise de décision.
J'espère que mes futurs apprentissages me permettront d'effectuer un choix éclairé et de trouver la voie professionnelle qui correspond le mieux à mes aspirations.
