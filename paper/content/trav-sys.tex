\section{Administration Système}
Le pôle administration système joue un rôle essentiel dans le maintien en condition opérationnelle des systèmes informatiques de l'entreprise.
Il est responsable de la gestion des sauvegardes et de la récupération des données, assurant ainsi la disponibilité et l'intégrité des informations cruciales.
De plus, il effectue également l'inventaire des équipements et des logiciels, ce qui permet de maintenir une visibilité précise sur les ressources technologiques de l'entreprise.
Grâce à ces responsabilités, le pôle administration système contribue à assurer la continuité des activités, la sécurité des données et la performance des systèmes, ce qui est essentiel pour la productivité et la réussite globale de l'organisation.

Le pôle d'administrateurs système est composé d'une seule personne, Delgutte Théo.
J'ai eu l'occasion de l'assister dans sa mission continue de maintien en condition opérationnelle de la partie informatique de l'entreprise.

\subsection{Infrastructure}
\subsubsection{serveur web pour tester la version php du wordpress et migratiions url grace au sql}
Suite à une modificationdes paramètres du serveur responsable pour l'hébergement du site vitrine d'amiral Technologies, la capacité des collaborateurs à l'éditer fut altérée.
La mission qui m'a été confiée fut de créer sur une machine virtuelle un serveur permettant d'héberger le site sur un domaine de développement pour pouvoir contrôler l'environnement utilisé.

\todo{?mettre nous}
Suite à des recherches sur internet, j'ai choisi de mettre en place un environnement composé des technologies suivantes : Linux, Nginx [prononcé engine-x], MariaBD et PHP ou LEMP en plus court.


\todo{J4AI FAIT DES REQUETES SQL}

\subsubsection{inventaire}
des pc ecrans et docks

des droits et applications

imagination d'un squelette pour une base de données d'inventaire - projet tombé à l'eau

adresses ip et schema réseau

\subsection{Cybersécurité}
\subsubsection{iso27001}
pourquoi : clients et cybersécu

récupération des documents de nécéssité pour l'ISO27001 : synthétisation

suivre des conference de la dga

création d'une présentation hebdomadaire sur le dernier mois sur les bonnes pratiques de sécu

\subsubsection{sophos ztna}
ajout d'une ressource 

prez outil employés

prez interface sysadmin pour intégration continue   

\subsubsection{wazuh}
serveur de monitoring

ajout d'une manière automatisée des agents sur les pc des employés à travers l'interface Intune

\subsubsection{docusaurus}
creation d'une documentation en ligne pour regrouper toutes les procédures

sur pc justine lol paske on m'a jamais donné de vm

création de multiples procédures employés

création de multiples procédures sysadmin
\newpage
\subsubsection{Docusaurus : Centralisation des Procédures}
Dans le but de simplifier l'accès aux procédures clés, nous avons créé une documentation en ligne en utilisant Docusaurus.
Cette plateforme nous permet de regrouper et de présenter de manière conviviale diverses procédures pour optimiser notre efficacité et garantir une expérience homogène.

Notre documentation, hébergée sur un PC \todo{reconditionné ou réadapté} exécutant Ubuntu, est accessible via une adresse IP locale, garantissant ainsi une accessibilité limitée aux collaborateurs.

Nous avons segmenté les procédures en deux volets :
\begin{itemize}
    \item \textbf{Procédures Employés :} Cette section couvre une variété d'outils et de situations pratiques, allant de l'utilisation de Google Drive à l'accueil d'un nouvel employé.
    Elle inclut également des astuces pour résoudre les problèmes informatiques courants et réagir en cas de suspicion de piratage.
    \item \textbf{Procédures Sysadmin :} Cette sous-section se penche sur les tâches spécifiques aux administrateurs système.
    Elle détaille des étapes telles que l'ajout de ressources via l'interface de Sophos, la gestion des agents et des ressources sur les PC des employés, ainsi que des instructions pour le nettoyage et le recyclage des appareils de stockage.
    De plus, elle propose des protocoles pour la création et la suppression de comptes d'employés, ainsi que des directives à suivre en cas d'attaque cyber.
\end{itemize}

En adoptant Docusaurus pour notre documentation, nous renforçons notre capacité à partager et à transmettre les connaissances internes de manière efficace, tout en améliorant la collaboration et la réactivité de notre équipe.
Cela garantit que chaque nouveau membre, qu'il soit employé ou administrateur système, dispose des ressources nécessaires pour contribuer à l'entreprise. 
\todo{[FAUX], à remplacer par une phrase plus vraie genre de disposer des ressources de base pour comprendre la contexte de base de l'utilisation et des règles de l'entreprise}


\subsubsection{L'isopérimètrie dans le monde de l'entreprise}
Dans le contexte professionnel actuel, équilibrer l'efficacité au travail et la cybersécurité est crucial.
Les entreprises font face à des menaces de cybersécurité de plus en plus sophistiquées, nécessitant une protection robuste.
Les attaques informatiques évoluent constamment, rendant difficile la prévention de chaque menace potentielle.

Les administrateurs système jouent un rôle central dans cet équilibre.
Responsables de la sécurité et de la performance des systèmes, ils doivent concevoir des environnements sûrs tout en facilitant la productivité.
Le défi réside dans la recherche du bon équilibre entre des mesures de sécurité strictes et des opérations fluides.

Cela exige des solutions astucieuses de la part des administrateurs système.
L'objectif ultime étant de créer un environnement où la sécurité est maximisée tout préservant la capacité à livrer de la valeur.

\subsection{Contact avec les collaborateurs}
phrase d'intro : on ne s'imaginerait pas que faire du sysadmin implique autant le contact humain 
\subsubsection{Sensibilisation}
prez hebdo

gophish:

- scrapping de sites officiels

- scrapping d'emails officiels

- lancement potentiels d'une campagne ?

\subsubsection{Assistance technique}
SAV et voider les pc

\subsubsection{Importance de la communication}
pk on a des pb into retirer cybersécu : communication aux choux

\subsubsection{changement de sysadmin}
infra changeante

gestionnaire de mdp de sysadmin

retiré le wazuh et sophos ztna
