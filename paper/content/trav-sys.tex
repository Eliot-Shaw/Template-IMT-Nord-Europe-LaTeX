\section{Administration système}
Le pôle administration système joue un rôle essentiel dans le maintien en condition opérationnelle des systèmes informatiques de l'entreprise.
Il est responsable de la gestion des sauvegardes et de la récupération des données, assurant ainsi la disponibilité et l'intégrité des informations cruciales.
De plus, il effectue également l'inventaire des équipements et des logiciels, ce qui permet de maintenir une visibilité précise sur les ressources technologiques de l'entreprise.
Grâce à ces responsabilités, le pôle administration système contribue à assurer la continuité des activités, la sécurité des données et la performance des systèmes, ce qui est essentiel pour la productivité et la réussite globale de l'organisation.

Le pôle d'administrateurs système est composé d'une seule personne, Delgutte Théo.
J'ai eu l'occasion de l'assister dans sa mission continue de maintien en condition opérationnelle de la partie informatique de l'entreprise.

\subsection{Objectifs initiaux}
\subsubsection{Réduction des risques}

\subsubsection{Attraction des clients}


\subsection{Missions effectuées}
\subsubsection{Gestion de l'infrastructure et de l'inventaire}
Dans le cadre de la gestion de l'infrastructure informatique au sein d'Amiral Technologies, j'ai entrepris des actions visant à améliorer la fonctionnalité, la stabilité et la gestion des différents composants de notre environnement.

\textbf{Serveur Web et Migration d'URL}

Une modification des paramètres du serveur hébergeant le site vitrine de l'entreprise a restreint la capacité des collaborateurs à effectuer des modifications.
Ma mission consistait alors à résoudre ce problème en mettant en place un serveur de développement dédié.
Cela permettrait de tester les changements sur un domaine de test avant de les appliquer au site en production.
Pour cela, j'ai choisi une configuration LEMP (Linux, Nginx, MariaDB, PHP).
J'ai également assuré la migration des URL en rédigeant des requêtes SQL pour adapter la base de données du site.

\textbf{Gestion de l'Inventaire Matériel et Utilisateurs}

Pour une meilleure gestion des ressources matérielles et des droits d'accès des utilisateurs, j'ai supervisé l'inventaire des ordinateurs, écrans et docks de l'entreprise.
J'ai également veillé à ce que les droits d'accès des utilisateurs aux applications soient en accord avec les politiques de sécurité en place.

\textbf{Conception d'une Base de Données d'Inventaire}

Pour centraliser l'ensemble des informations liées à l'inventaire, j'ai proposé la conception d'une base de données spécifique.
Cette structure aurait permis de suivre de manière systématique et précise les éléments matériels et les attributions aux utilisateurs.

\textbf{Optimisation du Schéma Réseau}

Afin d'améliorer la performance et l'efficacité de notre réseau, j'ai travaillé sur l'optimisation de son schéma.
Cela a inclus l'identification et la documentation minutieuse des adresses IP associées à chaque machine et service au sein de l'entreprise.

\subsubsection{Cyber}
L'approche de notre entreprise repose sur trois piliers essentiels : la recherche d'informations liées à la norme ISO27001, l'intégration de Sophos Zero Trust Network Access (ZTNA) et le déploiement de Wazuh pour la surveillance de sécurité.

En matière de sécurité, notre engagement envers la norme ISO27001 se traduit par la récupération et la synthétisation des documents clés.
De plus, j'ai eu l'opportunité de participer aux conférences de la Direction du Renseignement et de la Sécurité de la Défense, où j'ai pu échanger sur la situation de notre entreprise et la mise en place d'une nouvelle norme française de sécurité.

L'intégration de Sophos ZTNA reflète notre recherche constante de solutions avancées pour sécuriser nos systèmes et nos données.
L'agent Sophos ZTNA ne se contente pas de fournir une défense proactive, il intègre également un antivirus pour surveiller la santé de nos ordinateurs.
Les avantages englobent une gestion minutieuse des ressources, la sensibilisation des utilisateurs, une intégration harmonieuse dans nos flux de travail et une adhésion à une approche de confiance zéro.

En parallèle, notre engagement en matière de sécurité se matérialise par le déploiement de Wazuh, une solution de surveillance qui renforce la protection de notre infrastructure.
En centralisant la surveillance et l'indexation des journaux système au moyen d'un serveur Wazuh Manager et Wazuh Indexer, nous assurons une détection proactive des menaces.
En automatisant l'installation des agents Wazuh sur les ordinateurs via Intune, nous établissons une surveillance constante et fiable, nous permettant de réagir rapidement et efficacement.

Enfin, notre documentation prend une nouvelle forme grâce à Docusaurus, une plateforme qui centralise nos procédures pour une accessibilité et une efficacité accrues.
En hébergeant la documentation sur un PC dédié exécutant Ubuntu, nous garantissons un accès limité aux collaborateurs via une adresse IP locale.
En segmentant les procédures pour les employés et les administrateurs système, nous simplifions leur utilisation et contribuons à l'efficacité globale de notre équipe.
Cela renforce notre capacité à partager nos connaissances et à faciliter l'intégration fluide des nouveaux membres.


\subsubsection{iso27001}
pourquoi : clients et cybersécu

récupération des documents de nécéssité pour l'ISO27001 : synthétisation

suivre des conference de la dga

création d'une présentation hebdomadaire sur le dernier mois sur les bonnes pratiques de sécu

\subsubsection{Sophos Zero Trust Network Access (ZTNA)}

Dans la continuité de notre engagement envers une sécurité informatique robuste, nous avons intégré la solution Sophos Zero Trust Network Access (ZTNA).
Cette approche moderne de la sécurité réseau répond à nos besoins croissants en matière de protection des données et des systèmes, tout en favorisant la prévention proactive des menaces.
De plus, l'agent Sophos ZTNA comporte également un antivirus qui permet de surveiller la santé des ordinateurs et de fournir des informations essentielles à notre stratégie de sécurité.

\begin{itemize}
    \item \textbf{Gestion des Ressources Sécurisées :} Grâce à Sophos ZTNA, nous avons renforcé la sécurité de notre infrastructure en gérant de manière plus rigoureuse l'accès aux ressources internes.
    Cette fonctionnalité nous permet de limiter les accès non autorisés en fonction des utilisateurs, des appareils et des emplacements.
    \item \textbf{Sensibilisation des Utilisateurs :} Une présentation détaillée de ZTNA a été dispensée à nos employés pour les familiariser avec cet outil de sécurité.
    Cette initiative vise à promouvoir une utilisation responsable et sécurisée de la solution, renforçant ainsi la vigilance de nos équipes.
    \item \textbf{Intégration dans les Flux de Travail :} L'interface d'administration de ZTNA a été intégrée à nos processus de développement et d'intégration continue.
    Cela garantit que chaque nouveau système ou application est configuré en respectant nos normes de sécurité avant son déploiement.
    \item \textbf{Engagement envers la Confiance Zéro :} L'adoption de Sophos ZTNA témoigne de notre adhésion à une approche de sécurité axée sur la confiance zéro.
    Nous repensons la notion traditionnelle de confiance en privilégiant des contrôles d'accès détaillés basés sur l'identité et d'autres facteurs de risque.
\end{itemize}

\subsubsection{Sécurité Renforcée par Wazuh}

Nous avons déployé Wazuh, une solution de surveillance de sécurité, pour renforcer la protection de notre infrastructure informatique.
En utilisant un serveur Wazuh Manager et un Wazuh Indexer dans une machine virtuelle, nous centralisons la surveillance et l'indexation des journaux système.

Nous avons lié ce service à un sous-domaine de notre entreprise pour une identification précise des menaces.
De plus, en automatisant l'installation des agents Wazuh sur les ordinateurs des employés via Intune, nous assurons une surveillance cohérente et fiable.

Notre déploiement de Wazuh renforce la sécurité de notre infrastructure en simplifiant la surveillance et en automatisant le déploiement des agents.
Ainsi, Wazuh nous permet de réagir de manière proactive pour assurer la protection continue de notre entreprise.

\subsubsection{Docusaurus : Centralisation des Procédures}
Dans le but de simplifier l'accès aux procédures clés, nous avons créé une documentation en ligne en utilisant Docusaurus.
Cette plateforme nous permet de regrouper et de présenter de manière conviviale diverses procédures pour optimiser notre efficacité et garantir une expérience homogène.

Notre documentation, hébergée sur un PC \todo{reconditionné ou réadapté} exécutant Ubuntu, est accessible via une adresse IP locale, garantissant ainsi une accessibilité limitée aux collaborateurs.

Nous avons segmenté les procédures en deux volets :
\begin{itemize}
    \item \textbf{Procédures Employés :} Cette section couvre une variété d'outils et de situations pratiques, allant de l'utilisation de Google Drive à l'accueil d'un nouvel employé.
    Elle inclut également des astuces pour résoudre les problèmes informatiques courants et réagir en cas de suspicion de piratage.
    \item \textbf{Procédures Sysadmin :} Cette sous-section se penche sur les tâches spécifiques aux administrateurs système.
    Elle détaille des étapes telles que l'ajout de ressources via l'interface de Sophos, la gestion des agents et des ressources sur les PC des employés, ainsi que des instructions pour le nettoyage et le recyclage des appareils de stockage.
    De plus, elle propose des protocoles pour la création et la suppression de comptes d'employés, ainsi que des directives à suivre en cas d'attaque cyber.
\end{itemize}

En choisissant d'utiliser Docusaurus pour notre documentation, nous renforçons notre capacité à partager et à transmettre efficacement les connaissances internes au sein de notre équipe.
Cela assure que chaque nouveau membre, qu'il soit employé ou administrateur système, dispose des ressources de base nécessaires pour comprendre le contexte d'utilisation et les règles de l'entreprise.


\subsection{Retour et critiques sur le projet}
\subsubsection{Pas ou peu de réduction des risques et création de nouveaux risques}

\subsubsection{Perte de l'autonomie}


\subsection{Modification des objectifs}
\subsubsection{Eductation cyber}
prez hebdo

gophish:

- scrapping de sites officiels

- scrapping d'emails officiels

- lancement potentiels d'une campagne ?


\subsubsection{Mise en place d'une infrastructure alternative}
infra changeante --- retiré le wazuh et sophos ztna

\subsubsection{Modification des droits et autorisations}
gestionnaire de mdp de sysadmin


\subsection{Enseignements tirés}
\subsubsection{Importance de faire des choix éclairés}

\subsubsection{Importance de la communication}
pk on a des pb into retirer cybersécu : communication aux choux
phrase d'intro : on ne s'imaginerait pas que faire du sysadmin implique autant le contact humain 

\subsubsection{L'isopérimètrie dans le monde de l'entreprise}
Dans le contexte professionnel actuel, équilibrer l'efficacité au travail et la cybersécurité est crucial.
Les entreprises font face à des menaces de cybersécurité de plus en plus sophistiquées, nécessitant une protection robuste.
Les attaques informatiques évoluent constamment, rendant difficile la prévention de chaque menace potentielle.

Les administrateurs système jouent un rôle central dans cet équilibre.
Responsables de la sécurité et de la performance des systèmes, ils doivent concevoir des environnements sûrs tout en facilitant la productivité.
Le défi réside dans la recherche du bon équilibre entre des mesures de sécurité strictes et des opérations fluides.

Cela exige des solutions astucieuses de la part des administrateurs système.
L'objectif ultime étant de créer un environnement où la sécurité est maximisée tout préservant la capacité à livrer de la valeur.

\subsubsection{Identification des riques}