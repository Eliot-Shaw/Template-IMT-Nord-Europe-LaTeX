\section{travail effectué lors du stage}
\subsubsection{Méthode de travail}
Amiral Technologies fonctionne en mode agile, en suivant la méthode Scrum : l’entreprise travaille en sprint de 4 à 6 semaines.

Chaque sprint est composé de plusieurs épics (grandes fonctionnalités) regroupant différentes tâches d'un même thème.
À la fin d’un sprint, une réunion est organisée pour organisrer un feedback de la dernière période et d'observer l'avancée des projets.
Un jeu de société ou une activité est aussi organisée pour décompresser et souder l'équipe de développement pendant la réunion.

Des réunions trihebdomadaires sont également prévues pour créer une occasion de discuter des avancées immédiates et des potentiels bloquages que rencontrent les développeurs.
Ces réunions régulières permettent de génerer une aide et des idées au moment de la réunion d'une manière efficace.

\subsubsection{Environnement de travail}
Les bureaux d'Amiral Technologies sont agencés sous la forme d'un grand open space comprenant 20 bureaux et de deux salles de réunion.

Les horaires de bureau dépendent de la préférence de chacun : le consentius est de venir à 9h30 et de repartir entre 17h00 et 18h00.

Deux jours de télétravails sont alloués par semaine pour les employés le souhaitant, peronnellement j'optais pour travailler au bureau en effectif réduit.