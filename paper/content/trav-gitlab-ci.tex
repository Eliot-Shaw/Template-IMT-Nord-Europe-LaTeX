\section{Intégration continue}
\subsection{Technologies utilisées}
\subsubsection{Docker}
Docker est une plate-forme de virtualisation légère qui permet de créer, gérer et exécuter des applications dans des conteneurs.
Les conteneurs Docker encapsulent le code, les bibliothèques et les dépendances d'une application, garantissant une exécution cohérente et prévisible quel que soit l'environnement.
Contrairement aux machines virtuelles traditionnelles, les conteneurs partagent le noyau du système hôte, ce qui les rend plus efficaces en termes de ressources.

L'un des avantages clés de Docker est la portabilité, un seul fichier descriptif appelé dockerfile est respondable pour la création du contenaire.
Grâce à Docker, vous pouvez empaqueter votre application et toutes ses dépendances dans un conteneur unique, puis le déplacer sans effort entre différents environnements, qu'il s'agisse d'un ordinateur portable de développement, d'un serveur de production ou d'un cloud public.

Docker utilise des images pour définir le contenu et la configuration d'un conteneur.
Les images Docker sont des modèles immuables qui contiennent tout ce dont un conteneur a besoin pour s'exécuter.
Quand un fichier décrivant une image est modifié, seule la partie modifiée et ce qui suit est reconstruit, en réutilisant ce qui avait été précédemment construit.

\subsubsection{Docker Compose}
Docker Compose représente un outil simplifiant la gestion d'applications composées de plusieurs conteneurs.
Plutôt que de nécessiter une gestion manuelle de chaque conteneur et de ses paramètres respectifs, Docker Compose offre la possibilité de définir l'ensemble de la configuration d'une application au sein d'un fichier \texttt{docker-compose.yml}.

Dans ce fichier, les services sont spécifiés, lesquels constituent les divers composants de l'application, incluant leurs images Docker, les ports exposés, les variables d'environnement et d'autres options de configuration.
Par la suite, en utilisant la commande \texttt{docker-compose up}, il est possible de lancer simultanément tous les conteneurs, instaurant un environnement cohérent pour l'application.

De plus, Docker Compose simplifie la communication entre les conteneurs d'une application en mettant en place un réseau virtuel par défaut. Ceci permet aux services de se faire référence mutuellement par leur nom de service.

\subsubsection{Intégration Continue (CI) avec GitLab}
L'intégration continue (CI) est une pratique de développement logiciel visant à automatiser et à faciliter le processus de construction et de test d'une application.

La construction et le test de Diagfit se fait sur la plate-forme de gestion de développement GitLab avec une série de tâches automatisées, appelées "jobs", qui sont exécutées séquentiellement en réponse aux changements du code source.


\subsection{Besoin des developpeurs}
\subsubsection{architecture globale de Diagfit}
Les développeurs travaillaient sur la dernière version en production : Diagfit 2.5.

\todo{mettre drawio archi globale 2.5}

Ci-dessus, l'architecture globale de l'application Diagfit avec tous leur services.
Suite à une refactorisation de l'application, le système s'est grandement simplifié.

\todo{mettre drawio archi globale 2.6}

\subsubsection{Utilisation de Docker}
L'image docker qui était utilisée pour Diagfit lors de la CI demandait une certaine quantité de commandes pour définir l'environnement de l'application.
Reproduire le même environnement pour effectuer des tests était fastidieux et causait des différences entre l'environnement de test en local et celui produit lors de la CI par GitLab.

\subsubsection{Mission confiée}
La mission qui m'a été confiée fut d'optimiser le dockerfile et le fichier Docker Compose utilisés pour le logiciel Diagfit.

J'ai aussi eu l'occasion de créer une image Docker et un fichier Docker Compose pour créer un environnement contrôlé utilisable pour la création du logiciel et également pour les tests des développeurs.

\subsection{travail du refacto et rencontre des pb}
\subsubsection{Optimisation du dockerfile ci et docker compose}
La construction d'une image Docker à partir d'un dockerfile se fait ligne par ligne - étape par étape.
Lors de la reconstruction de l'image, Docker est capable de reprendre la construction au niveau de la première modification dans ce fichier.

L'initialisation de l'environnement de l'image Docker est le téléchargement des packages utilisés pour le logiciel étant long et peu changeant, il etait judicieux de les placer avant la récupération du code en développement.

La réduction des services utilisés nous a également permis de les retirer u fichier docker compose et donc d'accélérer le temps de lancement de l'application.

\subsubsection{Création de différents dockerfiles en fonction des utilisations}
La création d'un environnement identique entre la ci de Gitlab et l'ordinateur des développeurs backend était un point important dans la mission qui m'a été proposée.

J'ai pu créer une image spécifique pour le rendu de l'application et une autre pour le développement et les tests unitaires à partir des fichiers existants.

\subsubsection{Modification de Gitlab}
Sur la CI de Gitlab, on inclut le fichier dockerfile mais pas l'image construite de l'application.
Il nous a donc fallu lancer une image docker générique pour build l'image docker de développement et de test d'une manière automatisée.

Une des difficultés rencontrées lors de la modification de l'envoironnement Gitlab fut lors de l'implémentation du nouveau job dans la Ci de gitlab : certaines variables fixées au sein du dockerfile ne prenaient pas effet dans le job 


variables gitlab

- des variables ont été set dans l'environnement GitLab en variables globales contenant des informations sensibles

-> on les a enlevées et on a pu utiliser des variables personnalisées pour le programme

\subsection{rendu}
refacto du dockerfile

refacto du docker compose

refacto du gitlab-ci
    
- une image à récupérer pour faire les tests, plus besoin de la faire en local

- amélioration de la rapidité d'exectution 12'30" to 11'15" sur le job de test

- Robustesse du docker et de la CI !!!!
