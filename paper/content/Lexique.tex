\newglossaryentry{rootserver}{
    name={root name server}, 
    description={13 serveurs faisant autorité sur la racine d’internet et contenant donc les informations sur tous les TLD. Ils sont constitués de 1708 instances (serveur physiques) gérés par 12 opérateurs indépendants et répartis partout dans le monde.}, 
    plural={root name servers}*
}

\newglossaryentry{autoritativeserver}{
    name={serveur faisant autorité}, 
    description={Un serveur faisant autorité sur une zone DNS est un serveur dont les réponses concernant la zone sont considérées comme fiables à l’instant où elle est reçue. (Une réponse qui est gardée en mémoire n’est pas forcément à jour contrairement à une réponse venant directement du serveur faisant autorité)}, 
    plural={serveurs faisant autorité}
}

\newglossaryentry{tld}{
    name={tld},
    description={ou Top Level Domain (nom de domaine de premier niveau) est la partie a la fin d'un nom de domaine tel que le .com .net ou .fr}
}

\newglossaryentry{icann}{
    name={icann},
    description={ ou Internet Corporation for Assigned Names and Numbers est une société à but on lucratif et d'utilité publique consacré à la sécurité, la stabilité et l'interopérabilité de l'internet, elle est responsable de la coordination du système de nommage de l'internet et a donc un impact important sur l'évolution de l'internet}
}
\newglossaryentry{centr}{
    name={centr},
    description={ ou Concil of European National Top-Level domain Registries (Conseil des Registres de Premier Niveau Européens), c'est une association constitué de 52 Registres membres et 9 associés qui sont ensembles responsables de plus de 80\% des noms de domaine enregistrés dans le monde. Son objectif est de promouvoir et participer au développement de nouveau standards et meilleures pratiques}
}
\newglossaryentry{wsis}{
    name={wsis},
    description={ (World Summit on the Information Society), Sommet mondial sur la société de l'information (SMSI) en français est un forum mondial organisé par l'Union Internationale des Télécommunications, une agence de l'Organisation des Nations Unies (ONU). Il vise à réduire l'inégalité des habitants de la planète vis-à-vis de l'accès à l'information par le biais des nouvelles technologies de l'information et de la communication, en particulier d'Internet.}
}

\newglossaryentry{ietf}{
    name={ietf},
    description={ ou Internet Engineering Task Force est l'organisation responsable de la création des standards qui constituent les différents protocole de l'internet au travers des différents RFC}
}
\newglossaryentry{ripencc}{
    name={ripe ncc},
    description={ (Réseaux IP Européens - Network Coordination Centre), un registre régional d'adresses IP. Il dessert l'Europe et une partie de l'Asie, notamment au Moyen-Orient. Il est responsable de l'attribution des adresses IPv4, adresses IPv6,et des numéros d'Autonomous System (ASN) utilisés pour le routage entre opérateurs (utilisation du protocole BGP).}
}

\newglossaryentry{scrape}{
    name={scrape},
    description={processus consistant a extraire des données d'une page web}
}

\newglossaryentry{anycast}{
    name={nuage anycast},
    description={infrastructure internet consistant de plusieurs serveurs possédant la même adresse qui fait que le serveur le plus proche de l'utilisateur répond ce qui permet de grandement réduire le temps de réponse, la charge en bande passante et l'impact des attaques par déni de service ou DoS. (l'afnic et de nombreux autres organismes utilisent cette technologie depuis plus de 10 ans) \cite{anycastAFNIC} \cite{anycastRFC}} 
}
