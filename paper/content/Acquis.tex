\imtnechapitre{Impressions Personnelles et Acquis}
Lors de ce stage, j'ai découvert plusieurs outils très pratique notement Docker et Docker-compose qui est des solutions clefs en main qui permettent de déployer des programmes (site web, service, logiciel, autre) très facilement et indépendamment de la plateforme. Bien que ce soit pas recommandé pour des grosse infrastructures de production, il y a de nombreux cas où l'utilisation de Dockers est valide.

En dehors de cela, j'ai aussi apris a utilisé Prometheus et Grafana qui sont des outils très utilisés pour récupérer des données de monitoring de nombreuses sources et afficher ces données respectivement. De façon plus transverse, je me suis familiarisé avec des distribution de linux utilisés en production que je n'avais pas encore eu l'ocasion de manipuler tels que RedHat.

En sortant de l'aspect purement technique, j'ai pu développer ma prise de décision et mon autonomie car j'était responsable de faire les choix liés a ma mission (quels outils utiliser comment les implémenter, etc) et devais réfléchir au pour et au contre de chacun. Cela était très intéréssant car dans mes précédants stages j'était généralement restraint a l'infrastructure déjà en place et a des choix déjà faits car je me greffais à un projet déjà existant alors qu'ici je commançait un projet de zéro et c'est moi qui était responsable de faire mes recherches et apporter des solutions aux différents problèmes techniques.

Le mode de travail de l'équipe était aussi nouveau pour moi, le fait d'être dans une équipe où chaque personne travaille sur ses propre projets (certains gros projets avec 2 ou 3 personnes dessus maximum) a fait que j'ai apris et découvert beaucoup de choses lors des réunions d'équipe où beaucoup de sujets était abordés. Cela montre aussi encore plus l'importance d'avoir des documentations claires et complètes car il y en a beaucoup et ils ont généralement été réalisés par une seule personne.

J'ai aussi trouvé je rythme de travail 32h\verb|/|4jours meilleur que 35h\verb|/|5 jours car en comptant le temps perdu chaque jour a lire les emails, faire son planning de la journée et autre, il ne me semble pas avoir d'avoir d'impact conséquent sur le nombre d'heures effectives et le jour non travaillé par semaine est très utile, en dehors du temps libre en plus, c'est aussi l'occasion de faires ses démarches administratives et autres auprès des organismes qui ne sont pas ouvert le weekend.

Ce stage m'a donc permis d'avancer sur mon projet professionnel et m'a conforté dans ma décision de m'orienter vers la partie système/réseau/sécurité de l'informatique et non sur la partie développement pur.
